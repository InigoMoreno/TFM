%
% Introduction
\chapter{Introduction} \label{chap::intro}

\section{Theme Relevance and Justification}

\subsection{Omni-wheeled robots}

\subsection{Model Predictive Control}

\section{Research Objective}
The objective of the research included in this thesis is to:

\begin{quote}
	\emph{
		Design a control algorithm for systems with an omni-wheeled robot and demonstrate it in simulated and real scenarios.
	}
\end{quote}

\section{Thesis Context}

This thesis is a result of the collaboration between \ac{FME} and \ac{IRI}. During the summer of 2018 I contacted \mscsupervisorone looking for a suitable project for my Master's Thesis and he proposed me this one.

\section{Notation}

Throughout this thesis the following notation will be used:

\setlength\arrayrulewidth{1.2pt}
\let\oldarraystretch\arraystretch
\renewcommand{\arraystretch}{1.2}
%\rowcolors{2}{gray!10}{white}
\begin{table}[H]
\begin{center}
\begin{tabular}{ l | c | c }
	& Symbol & Definition \\\hline
	Scalars: & $x$ & \\ 
	Vectors: & $\vec{x}$ & \\  
	Elements of vector: & $\vec{x}(i)$ & $i$-th element of vector $\vec x$\\ 
	Position vectors: & $\vec p_x$ & \\
	Coordinates of position vector: &$x_x, y_x, z_x$&x,y,z coordinates of position $\vec p_x$\\
	Matrices: & $\vec{X}$ & \\
	Sets: & $\mathcal{X}$ & \\
	%Set of unit vectors: & $\mathcal{S}^{n-1}$ & $\{\vec{x}\in \R^n | \ \|\vec{x}\|=1\}$ \\
	Time derivatives: & $\dot{x}$ & $\frac{\partial x}{\partial t}$ \\
	Desired values: & $\bar{x} $&  desired value of $x$ \\
	Predicted values: & $\hat{x} $&  predicted value of $x$ 
	%Set of positive semidefinite matrices of size $n$: & $\mathcal{S}^n_+$ & $\{\vec{A\in\R^{n\times n}}| \ \vec{x}^T\vec{Ax}>0 \ \forall \ x\in \R_{\ne 0}^n\}$\\ 
	%Weighted squared norm: & $\|\vec{x}\|_{\vec{Q}}$ & $\vec{x}^T\vec{Qx} \ \forall \  \vec{x}\in\R^n ,\ \vec{Q}\in \mathcal{S}^n_+ $ \\
%	Hat map: & $\hat{x}$ & $\hat{x} y=x\times y \ \forall \ x,y\in\R^3$
	\label{tab:Notation}
\end{tabular}
\caption{Thesis Notation}
\end{center}
\end{table}
All vectors are column vectors and all positions are expressed in the \ac{ENU} inertial frame unless otherwise stated.




\let\arraystretch\oldarraystretch



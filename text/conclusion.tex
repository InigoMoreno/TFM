%
% Conclusion
\chapter{Conclusion} \label{chap::conclusion}
In this thesis we introduced a new approach to the control of multiple \acp{UAV} carrying a slung payload through wires. This controller can be adapted to a variable number of quadrotors, obstacles and different environments. It is an agile controller able to generate complex maneuvers on the go even with uncertainties. It is able to dynamically adapt to changing environments, such as unexpected movement of obstacles or change of objective. The dynamic model is free of singularities, takes into account aerodynamic drag. Constraints prevent the system from doing unsafe movements and avoid wire slackening. The controller has been demonstrated to work in real-time simulated environments and the experiments on real drones, although unsuccessful, look promising.

\section{Future Studies}
For future studies we recommend the following:
\begin{itemize}
	\item Do more experimental studies of the controller.
	\item Test the system with a wider variety of parameters, such as heavier payloads or longer strings.
	\item For control of systems with more than three drones, it is almost impossible to prevent some of the wires from slackening, therefore a dynamic model that includes wire slackening would be preferred. Maybe a model where each of the wires can switch modes such as in \cite{Bisgaard2009}
	\item A more realistic model of the strings, where their mass is not neglected and they act more as springs rather than rigid cables would improve the accuracy of the dynamic equations.
	\item To improve realism and maybe fix the problems described in \cref{sect::experimental} it would be interesting to describe the string attachment points on the drone and on the payload.
	\item When the payload gets heavier, it would also be necessary to model it as a rigid body instead of as a point mass. This is done in \cite{Lee2014}.
	\item The complex dynamic equations could be substituted by some artificial intelligence which mimics the dynamic movement of the system.
	\item Drones with more direct inputs could be used so that we would not need to generate models of the internal dynamics.
	\item It would be ideal to switch the \ac{MPC} solver to one which can be limited by solve time instead of by the number of solver iterations.	
	\item An on-board control could be tested to rid the controller from the need of a \ac{MCS}.
	\item De-centralized control could be attempted for better scalability to an increase in the number of drones.
\end{itemize}
